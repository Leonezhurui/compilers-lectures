% dfa-lexer.tex

%%%%%%%%%%%%%%%%%%%%
\begin{frame}{}
  \fig{width = 0.60\textwidth}{figs/dfa-lexer}
  \begin{center}
    自动化词法分析器
  \end{center}
\end{frame}
%%%%%%%%%%%%%%%%%%%%

%%%%%%%%%%%%%%%%%%%%
\begin{frame}{}
  \begin{center}
    \blue{\large 自动机} \\[5pt]
    (Automaton; Automata)

    \fig{width = 0.50\textwidth}{figs/off-on}
    \vspace{-0.30cm}
    ``开关''自动机

    \vspace{0.80cm}
    两大要素: {\bf 状态集} $S$ 以及{\bf 状态转移函数} $\delta$
  \end{center}
\end{frame}
%%%%%%%%%%%%%%%%%%%%

%%%%%%%%%%%%%%%%%%%%
\begin{frame}{}
  \begin{center}
    \fig{width = 0.60\textwidth}{figs/automata-theory-wiki}

    \vspace{0.50cm}
    根据\red{\bf 表达/计算能力}的强弱, 自动机可以分为不同层次。
  \end{center}
\end{frame}
%%%%%%%%%%%%%%%%%%%%

%%%%%%%%%%%%%%%%%%%%
\begin{frame}{}
  \begin{center}
    元胞自动机 (Cellular Automaton)
  \end{center}

  \vspace{0.30cm}
  \begin{columns}
    \column{0.60\textwidth}
      \fig{width = 1.00\textwidth}{figs/breeder}
      \begin{center}
        ``播种机''、``滑翔机枪''与``滑翔机'' \\[5pt]
      \end{center}
    \column{0.40\textwidth}
      \fig{width = 0.90\textwidth}{figs/John-Conway}
      \begin{center}
        \teal{John Horton Conway} \\[3pt]
        \teal{(1937 $\sim$ 2020)}
      \end{center}
  \end{columns}

  \begin{center}
    \teal{\url{https://en.wikipedia.org/wiki/File:Conways_game_of_life_breeder_animation.gif}}
  \end{center}
\end{frame}
%%%%%%%%%%%%%%%%%%%%

%%%%%%%%%%%%%%%%%%%%
\begin{frame}{}
  \begin{center}
    \fig{width = 0.40\textwidth}{figs/video}

    \vspace{0.80cm}
    ``生命游戏''(Game of Life)史诗级巨作 \\[8pt]
    \teal{\url{https://www.youtube.com/watch?v=C2vgICfQawE&t=270s}}
  \end{center}
\end{frame}
%%%%%%%%%%%%%%%%%%%%

%%%%%%%%%%%%%%%%%%%%
\begin{frame}{}
  \begin{center}
    \red{\bf 目标: 正则表达式 RE $\implies$ 词法分析器}

    \vspace{0.30cm}
    \fig{width = 0.70\textwidth}{figs/re-dfa-lexer}

    \vspace{0.50cm}
    \purple{终点固然令人向往, 这一路上的风景更是美不胜收}
  \end{center}
\end{frame}
%%%%%%%%%%%%%%%%%%%%

%%%%%%%%%%%%%%%%%%%%
\begin{frame}{}
  \begin{definition}[NFA (Nondeteministic Finite Automaton)]
    非确定性有穷自动机 $\mathcal{A}$ 是一个五元组 
    \red{$\mathcal{A} = (\Sigma, S, s_{0}, \delta, F)$}:

    \vspace{0.30cm}
    \begin{enumerate}[(1)]
      \item 字母表 $\Sigma$ ($\epsilon \notin \Sigma$)
      \item \red{\bf 有穷}的状态集合 $S$
      \item \purple{唯一}的初始状态 $s_{0} \in S$
      \item 状态转移\red{\bf 函数} $\delta$
        \[
          \delta: S \times (\Sigma \cup \set{\blue{\epsilon}}) \to \blue{2^{S}}
        \]
      \item 接受状态集合 $F \subseteq S$
    \end{enumerate}
  \end{definition}

  \fig{width = 0.60\textwidth}{figs/nfa-abb}

  \pause
  \begin{center}
    \purple{\bf 约定:} 所有没有对应出边的字符默认指向一个不存在的{\bf ``空状态'' $\emptyset$}
  \end{center}
\end{frame}
%%%%%%%%%%%%%%%%%%%%

%%%%%%%%%%%%%%%%%%%%
\begin{frame}{}
  \begin{columns}
    \column{0.33\textwidth}
      \fig{width = 0.55\textwidth}{figs/rabin}
      \begin{center}
        \teal{Michael O. Rabin (1931 $\sim$)}
      \end{center}
    \column{0.34\textwidth}
      \fig{width = 1.20\textwidth}{figs/fa-paper}
      \begin{center}
        发表于 1959 年; \\[6pt]
        1976年, 共享图灵奖 \\
      \end{center}
    \column{0.33\textwidth}
      \fig{width = 0.60\textwidth}{figs/scott}
      \begin{center}
        \teal{Dana Scott (1932 $\sim$)}
      \end{center}
  \end{columns}

  \vspace{0.60cm}
  \begin{center}
    \begin{quote}
    ``which introduced the idea of \red{\emph{\textbf{nondeterministic machines}}}, \\[6pt]
      which has proved to be an enormously valuable concept.''
    \end{quote}
  \end{center}
\end{frame}
%%%%%%%%%%%%%%%%%%%%
\begin{frame}{}
  \begin{center}
    (非确定性)有穷自动机是一类极其简单的\red{\bf 计算}装置

    \vspace{0.60cm}
    它可以\red{\bf 识别} (接受/拒绝) $\Sigma$ 上的字符串

    \vspace{0.50cm}
    \begin{definition}[接受 (Accept)]
      (非确定性)有穷自动机 $\mathcal{A}$ 接受字符串$x$,
      当且仅当\red{\bf 存在}一条从开始状态 $s_{0}$ 到\red{\bf 某个}接受状态 $f \in F$、
      标号为 $x$ 的路径。
    \end{definition}

    \vspace{0.80cm}
    因此, \purple{$\mathcal{A}$ 定义了一种{\bf 语言} $L(\mathcal{A})$}: 
    它能接受的所有字符串构成的集合
  \end{center}
\end{frame}
%%%%%%%%%%%%%%%%%%%%

%%%%%%%%%%%%%%%%%%%%
\begin{frame}{}
  \fig{width = 0.70\textwidth}{figs/nfa-abb}

  \[
    aabb \in L(\mathcal{A}) \qquad ababab \notin L(\mathcal{A})
  \]

  \pause
  \[
    L(\mathcal{A}) = \pause L((a|b)^{\ast}abb)
  \]
\end{frame}
%%%%%%%%%%%%%%%%%%%%

%%%%%%%%%%%%%%%%%%%%
\begin{frame}{}
  \begin{center}
    关于自动机 $\mathcal{A}$ 的\red{\bf 两个基本问题}:
  \end{center}

  \vspace{0.60cm}
  \begin{columns}
    \column{0.10\textwidth}
    \column{0.80\textwidth}
      \begin{itemize}
        \setlength{\itemsep}{15pt}
        \item \blue{\bf Membership 问题:} 
          \text{给定字符串 $x$}, $x \in L(\mathcal{A})$?
        \item \blue{$L(\mathcal{A})$} 究竟是什么?
      \end{itemize}
    \column{0.10\textwidth}
  \end{columns}
\end{frame}
%%%%%%%%%%%%%%%%%%%%

%%%%%%%%%%%%%%%%%%%%
\begin{frame}{}
  \fig{width = 0.45\textwidth}{figs/nfa-aa-bb}
  \[
    aaa \in \mathcal{A}? \qquad aab \in \mathcal{A}?
  \]

  \[
    \red{L(\mathcal{A})} = \pause L((aa^{\ast}|bb^{\ast}))
  \]
\end{frame}
%%%%%%%%%%%%%%%%%%%%

%%%%%%%%%%%%%%%%%%%%
\begin{frame}{}
  \fig{width = 0.50\textwidth}{figs/nfa-even-01}

  \vspace{-0.50cm}
  \[
    1011 \in L(\mathcal{A})? \qquad 0011 \in L(\mathcal{A})?
  \]

  \pause
  \vspace{-0.30cm}
  \[
    L(\mathcal{A}) = \set{\text{包含偶数个1或偶数个0的01串}}
  \]
\end{frame}
%%%%%%%%%%%%%%%%%%%%

%%%%%%%%%%%%%%%%%%%%
\begin{frame}{}
  \begin{definition}[DFA (Deterministic Finite Automaton)]
    确定性有穷自动机 $\mathcal{A}$ 是一个五元组 
    \red{$\mathcal{A} = (\Sigma, S, s_{0}, \delta, F)$}:

    \vspace{0.30cm}
    \begin{enumerate}[(1)]
      \item 字母表 $\Sigma$ ($\epsilon \notin \Sigma$)
      \item \red{\bf 有穷}的状态集合 $S$
      \item \red{\bf 唯一}的初始状态 $s_{0} \in S$
      \item 状态转移\red{\bf 函数} $\delta$
        \[
          \delta: S \times \blue{\Sigma} \to \blue{S}
        \]
      \item 接受状态集合 $F \subseteq S$
    \end{enumerate}
  \end{definition}

  \fig{width = 0.50\textwidth}{figs/dfa-abb}

  \pause
  \begin{center}
    \purple{\bf 约定:} 所有没有对应出边的字符默认指向一个不存在的{\bf ``死状态''}
  \end{center}
\end{frame}
%%%%%%%%%%%%%%%%%%%%

%%%%%%%%%%%%%%%%%%%%
\begin{frame}{}
  \fig{width = 0.50\textwidth}{figs/dfa-abb}

  \[
    aabb \in L(\mathcal{A}) \qquad ababab \notin L(\mathcal{A})
  \]

  \pause
  \[
    L(\mathcal{A}) = \pause L((a|b)^{\ast}abb)
  \]

  \pause
  \fig{width = 0.50\textwidth}{figs/nfa-abb}
\end{frame}
%%%%%%%%%%%%%%%%%%%%

%%%%%%%%%%%%%%%%%%%%
\begin{frame}{}
  \begin{center}
    NFA 简洁易于理解, 方面描述语言 \blue{$L(\mathcal{A})$}

    \vspace{0.30cm}
    DFA 易于判断 \blue{$x \in L(\mathcal{A})$}, 适合产生词法分析器 

    \pause
    \vspace{1.20cm}
    用 NFA 描述语言, 用 DFA 实现词法分析器

    \vspace{0.30cm}
    \purple{RE $\implies$ NFA $\implies$ DFA $\implies$ 词法分析器}
  \end{center}
\end{frame}
%%%%%%%%%%%%%%%%%%%%

%%%%%%%%%%%%%%%%%%%%
\begin{frame}{}
  \fig{width = 0.70\textwidth}{figs/re-dfa-lexer}

  \vspace{0.30cm}
  \begin{center}
    \blue{RE $\implies$ NFA}
    \[
      \boxed{r \implies N(r)}
    \]
    \[
      \text{要求}: \red{L(N(r)) = L(r)}
    \]
  \end{center}
\end{frame}
%%%%%%%%%%%%%%%%%%%%

%%%%%%%%%%%%%%%%%%%%
\begin{frame}{}
  \begin{center}
    从 RE 到 NFA: \red{\bf Thompson 构造法}
  \end{center}
  
  \pause
  \vspace{0.20cm}
  \fig{width = 0.60\textwidth}{figs/Thompson-Ritchie-Unix}
  \begin{center}
    Turing Award, 1983
  \end{center}

  % \begin{columns}
  %   \column{0.50\textwidth}
  %     \fig{width = 0.50\textwidth}{figs/Thompson}
  %     \begin{center}
  %       Ken Thompson (1943 $\sim$)
  %     \end{center}
  %   \column{0.50\textwidth}
  %     \fig{width = 0.50\textwidth}{figs/Ritchie}
  %     \begin{center}
  %       Dennis Ritchie (1941 $\sim$ 2011)
  %     \end{center}
  % \end{columns}
  
  % \begin{quote}
  %   \centering
  %   {\small ``For their development of generic operating systems theory \\
  %   and specifically for the implementation of the \red{UNIX} operating system.''
  %   \hfill --- Turing Award, 1983}
  % \end{quote}
\end{frame}
%%%%%%%%%%%%%%%%%%%%

%%%%%%%%%%%%%%%%%%%%
\begin{frame}{}
  \begin{center}
    Thompson 构造法的基本思想: \red{\bf 按结构归纳}
  \end{center}

  \begin{definition}[正则表达式]
    给定字母表 $\Sigma$, $\Sigma$ 上的正则表达式\red{\bf 由且仅由}以下规则定义:
    \begin{enumerate}[(1)]
      \item $\epsilon$ 是正则表达式;
      \item $\forall a \in \Sigma$, $a$ 是正则表达式;
      \item 如果 $s$ 是正则表达式, 则 $(s)$ 是正则表达式;
      \item 如果 $s$ 与 $t$ 是正则表达式, 则 $s|t$, $st$, $s^{\ast}$ 也是正则表达式。
    \end{enumerate}
  \end{definition}
\end{frame}
%%%%%%%%%%%%%%%%%%%%

%%%%%%%%%%%%%%%%%%%%
\begin{frame}{}
  \begin{center}
    $\epsilon$ 是正则表达式。

    \pause
    \vspace{0.80cm}
    \fig{width = 0.40\textwidth}{figs/epsilon-nfa}
  \end{center}
\end{frame}
%%%%%%%%%%%%%%%%%%%%

%%%%%%%%%%%%%%%%%%%%
\begin{frame}{}
  \begin{center}
    $a \in \Sigma$ 是正则表达式。

    \pause
    \vspace{0.80cm}
    \fig{width = 0.40\textwidth}{figs/symbol-nfa}
  \end{center}
\end{frame}
%%%%%%%%%%%%%%%%%%%%

%%%%%%%%%%%%%%%%%%%%
\begin{frame}{}
  \begin{center}
    如果 $s$ 是正则表达式, 则 $(s)$ 是正则表达式;

    \pause
    \vspace{0.60cm}
    \[
      N((s)) = N(s).
    \]
  \end{center}
\end{frame}
%%%%%%%%%%%%%%%%%%%%

%%%%%%%%%%%%%%%%%%%%
\begin{frame}{}
  \begin{center}
    如果 $s$, $t$ 是正则表达式, 则 $s | t$ 是正则表达式。

    \pause
    \vspace{0.50cm}
    \fig{width = 0.50\textwidth}{figs/or-nfa}

    \pause
    \vspace{0.50cm}
    \red{$Q:$ 如果 $N(s)$ 或 $N(t)$ 的开始状态或接受状态不唯一, 怎么办?}

    \pause
    \vspace{0.50cm}
    根据\blue{\bf 归纳假设}, $N(s)$ 与 $N(t)$ 的开始状态与接受状态均\blue{\bf 唯一}。
  \end{center}
\end{frame}
%%%%%%%%%%%%%%%%%%%%

%%%%%%%%%%%%%%%%%%%%
\begin{frame}{}
  \begin{center}
    如果 $s$, $t$ 是正则表达式, 则 $st$ 是正则表达式。

    \pause
    \vspace{0.50cm}
    \fig{width = 0.60\textwidth}{figs/concat-nfa}

    \pause
    \vspace{0.50cm}
    根据\blue{\bf 归纳假设}, $N(s)$ 与 $N(t)$ 的开始状态与接受状态均\blue{\bf 唯一}。
  \end{center}
\end{frame}
%%%%%%%%%%%%%%%%%%%%

%%%%%%%%%%%%%%%%%%%%
\begin{frame}{}
  \begin{center}
    如果 $s$ 是正则表达式, 则 $s^{\ast}$ 是正则表达式。

    \pause
    \vspace{0.50cm}
    \fig{width = 0.60\textwidth}{figs/kleene-star-nfa}

    \pause
    \vspace{0.50cm}
    根据\blue{\bf 归纳假设}, $N(s)$ 的开始状态与接受状态\blue{\bf 唯一}。
  \end{center}
\end{frame}
%%%%%%%%%%%%%%%%%%%%

%%%%%%%%%%%%%%%%%%%%
\begin{frame}{}
  \begin{center}
    $N(r)$ 的\red{\bf 性质}以及 Thompson 构造法\red{\bf 复杂度分析}
  \end{center}

  \vspace{0.30cm}
  \begin{columns}
    \column{0.10\textwidth}
    \column{0.80\textwidth}
      \begin{enumerate}
        \setlength{\itemsep}{8pt}
        \item $N(r)$ 的开始状态与接受状态均\blue{\bf 唯一}。
        \item 开始状态没有入边, 接受状态没有出边。
        \pause
        \item $N(r)$ 的\blue{\bf 状态数} $|S| \le$ $2 \times |r|$。\\
          ($|r|:$ $r$ 中运算符与运算分量的总和)
        \pause
        \item 每个状态最多有两个$\epsilon$-入边与两个$\epsilon$-出边。
        \item $\forall a \in \Sigma$, 每个状态最多有一个$a$-入边与一个$a$-出边。
      \end{enumerate}
    \column{0.10\textwidth}
  \end{columns}
\end{frame}
%%%%%%%%%%%%%%%%%%%%

%%%%%%%%%%%%%%%%%%%%
\begin{frame}{}
  \[
    r = (a | b)^{\ast} abb
  \]

  \pause
  \begin{columns}
    \column{0.30\textwidth}
      \fig{width = 0.80\textwidth}{figs/r1-nfa-ex}
    \column{0.30\textwidth}
      \fig{width = 0.80\textwidth}{figs/r2-nfa-ex}
    \column{0.40\textwidth}
      \pause
      \fig{width = 0.95\textwidth}{figs/r3-nfa-ex}
  \end{columns}

  \begin{columns}
    \column{0.45\textwidth}
      \pause
      \fig{width = 0.95\textwidth}{figs/r5-nfa-ex}
    \column{0.55\textwidth}
      \pause
      \fig{width = 1.00\textwidth}{figs/re-nfa-abb}
  \end{columns}
\end{frame}
%%%%%%%%%%%%%%%%%%%%

%%%%%%%%%%%%%%%%%%%%
\begin{frame}{}
  \fig{width = 0.70\textwidth}{figs/re-dfa-lexer}
  
  \vspace{0.30cm}
  \begin{center}
    \blue{NFA $\implies$ DFA}
    \[
      \boxed{N \implies D}
    \]
    \[
      \text{要求}: \red{L(D) = L(N)}
    \]
  \end{center}
\end{frame}
%%%%%%%%%%%%%%%%%%%%

%%%%%%%%%%%%%%%%%%%%
\begin{frame}{}
  \begin{center}
    从 NFA 到 DFA 的转换: \red{\bf 子集构造法} (Subset/Powerset Construction)

    \vspace{0.50cm}
    \begin{columns}
      \column{0.50\textwidth}
        \fig{width = 0.45\textwidth}{figs/rabin}
        \begin{center}
          \teal{Michael O. Rabin (1931 $\sim$)}
        \end{center}
      \column{0.50\textwidth}
        \fig{width = 0.50\textwidth}{figs/scott}
        \begin{center}
          \teal{Dana Scott (1932 $\sim$)}
        \end{center}
    \end{columns}
  \end{center}

  \vspace{0.20cm}
  \begin{center}
    \blue{\bf 思想: 用 DFA 模拟 NFA}
  \end{center}
\end{frame}
%%%%%%%%%%%%%%%%%%%%

%%%%%%%%%%%%%%%%%%%%
% \begin{frame}{}
%   \begin{center}
%     \red{\bf 用 DFA 模拟 NFA}
% 
%     \fig{width = 0.50\textwidth}{figs/nfa-abb}
%     \[
%       \blue{L(\mathcal{A}) = L((a|b)^{\ast}abb)}
%     \]
%     \fig{width = 0.50\textwidth}{figs/dfa-abb}
%   \end{center}
% \end{frame}
%%%%%%%%%%%%%%%%%%%%

%%%%%%%%%%%%%%%%%%%%
\begin{frame}{}
  \begin{center}
    \red{\bf 用 DFA 模拟 NFA}

    \fig{width = 0.75\textwidth}{figs/nfa-abb-epsilon}
    \vspace{-0.20cm}
    \[
      \blue{L(\mathcal{A}) = L((a|b)^{\ast}abb)}
    \]
    \vspace{-0.30cm}
    \fig{width = 0.50\textwidth}{figs/dfa-abb-subset}
  \end{center}
\end{frame}
%%%%%%%%%%%%%%%%%%%%

% %%%%%%%%%%%%%%%%%%%%
% \begin{frame}{}
%   \fig{width = 0.90\textwidth}{figs/re-nfa-abb-A}
%   \[
%     A \triangleq \epsilon\text{-closure}(0) = \set{0, 1, 2, 4, 7}
%   \]
% 
%   \begin{center}
%     DFA $D$ 的\purple{\bf 初始状态} $d_{0} = \epsilon\text{-closure}(n_{0})$
%   \end{center}
% \end{frame}
% %%%%%%%%%%%%%%%%%%%%
% 
% %%%%%%%%%%%%%%%%%%%%
% \begin{frame}{}
%   \fig{width = 0.90\textwidth}{figs/re-nfa-abb-B}
% 
%   \[
%     A = \set{0, 1, 2, 4, 7}
%   \]
%   \[
%     B \triangleq \delta_{D}(A, a) = \epsilon\text{-closure}(\set{3, 8}) 
%     = \set{1, 2, 3, 4, 6, 7, 8}
%   \]
% \end{frame}
% %%%%%%%%%%%%%%%%%%%%
% 
% %%%%%%%%%%%%%%%%%%%%
% \begin{frame}{}
%   \fig{width = 0.90\textwidth}{figs/re-nfa-abb-C}
% 
%   \[
%     A = \set{0, 1, 2, 4, 7}
%   \]
%   \[
%     C \triangleq \delta_{D}(A, b) = \epsilon\text{-closure}(5) 
%     = \set{1, 2, 4, 5, 6, 7}
%   \]
% \end{frame}
% %%%%%%%%%%%%%%%%%%%%
% 
% %%%%%%%%%%%%%%%%%%%%
% \begin{frame}{}
% \end{frame}
% %%%%%%%%%%%%%%%%%%%%
% 
% %%%%%%%%%%%%%%%%%%%%
% \begin{frame}{}
%   \fig{width = 0.90\textwidth}{figs/re-nfa-abb-D}
% 
%   \[
%     B = \set{1, 2, 4, 5, 6, 7}
%   \]
%   \[
%     D \triangleq \delta_{D}(B, b) = \epsilon\text{-closure}(\set{5, 9}) 
%     = \set{1, 2, 4, 5, 6, 7, 9}
%   \]
% \end{frame}
% %%%%%%%%%%%%%%%%%%%%

%%%%%%%%%%%%%%%%%%%%
\begin{frame}{}
  \begin{center}
    \fig{width = 0.40\textwidth}{figs/blackboard}

    \vspace{0.30cm}
    板书演示算法过程
  \end{center}
\end{frame}
%%%%%%%%%%%%%%%%%%%%

%%%%%%%%%%%%%%%%%%%%
\begin{frame}{}
  \fig{width = 0.50\textwidth}{figs/nfa-dfa-table}
  \fig{width = 0.60\textwidth}{figs/dfa-abb-subset}
\end{frame}
%%%%%%%%%%%%%%%%%%%%

%%%%%%%%%%%%%%%%%%%%
\begin{frame}{}
  \begin{center}
    从状态 $s$ 开始, 只通过 $\epsilon$-转移可达的状态集合
    \[
      \epsilon\text{-closure}(s) = \set{t \in S_{N} | s \rel{\epsilon^{\ast}} t}
    \]

    \pause
    \vspace{0.30cm}
    \[
      \epsilon\text{-closure}(T) = \bigcup_{s \in T} \epsilon\text{-closure}(s)
    \]

    \pause
    \vspace{0.30cm}
    \[
      \text{move}(T, a) = \bigcup_{s \in T} \delta(s, a)
    \]
  \end{center}
\end{frame}
%%%%%%%%%%%%%%%%%%%%

%%%%%%%%%%%%%%%%%%%%
\begin{frame}{}
  \begin{center}
    子集构造法 ($N \implies D$) 的\red{\bf 原理}: 
  \end{center}
  \[
    \blue{N}: (\Sigma_{N}, S_{N}, n_{0}, \delta_{N}, F_{N})
  \]
  \[
    \red{D}: (\Sigma_{D}, S_{D}, d_{0}, \delta_{D}, F_{D})
  \]

  \pause
  \[
    \Sigma_{D} = \Sigma_{N}
  \]

  \pause
  \[
    S_{D} \subseteq 2^{S_{N}} \qquad (\forall s_{D} \in S_{D}: \blue{s_{D} \subseteq S_{N}})
  \]

  \pause
  \[
    \text{\purple{\bf 初始状态}}\;
    {d_{0} = \epsilon\text{-closure}(n_{0})}
  \]
  \pause
  \vspace{-0.30cm}
  \[
    \text{\purple{\bf 转移函数}}\; 
    \forall a \in \Sigma_{D}: 
    {\delta_{D}(s_{D}, a) = \epsilon\text{-closure}(\text{move}(s_{D}, a))}
  \]
  \pause
  \vspace{-0.30cm}
  \[
    \text{\purple{\bf 接受状态集}}\;
    F_{\mathcal{D}} = \set{s_{D} \in S_{\mathcal{D}} \mid 
      \exists f \in F_{\mathcal{N}}.\; f \in s_{D}}
  \]
\end{frame}
%%%%%%%%%%%%%%%%%%%%

%%%%%%%%%%%%%%%%%%%%
\begin{frame}{}
  \begin{center}
    消除\red{\bf ``死状态 $\emptyset$''}

    \[
      \text{move}(T, a) = \emptyset
        \implies \delta(T, a) = \emptyset
    \]

    \[
      \forall a \in \Sigma.\; \delta(\emptyset, a) = \emptyset
    \]

    需要消除死状态, 避免词法分析器进入$\emptyset$状态
  \end{center}
\end{frame}
%%%%%%%%%%%%%%%%%%%%

%%%%%%%%%%%%%%%%%%%%
\begin{frame}{}
  \begin{center}
    子集构造法 ($N \implies D$) 的\red{\bf 实现}: 
    使用\blue{\bf 栈}实现\blue{$\epsilon\text{-closure}(T)$}

    \fig{width = 0.90\textwidth}{figs/epsilon-closure-impl}
  \end{center}
\end{frame}
%%%%%%%%%%%%%%%%%%%%

%%%%%%%%%%%%%%%%%%%%
\begin{frame}{}
  \begin{center}
    子集构造法 ($N \implies D$) 的\red{\bf 实现}: 
    使用\blue{\bf 标记搜索}过程构造\blue{\bf 状态集}

    \fig{width = 0.90\textwidth}{figs/subset-construction}
  \end{center}
\end{frame}
%%%%%%%%%%%%%%%%%%%%

%%%%%%%%%%%%%%%%%%%%
\begin{frame}{}
  \begin{center}
    子集构造法的\red{\bf 复杂度分析}: \\
    ($|S_{N}| = n$)

    \[
      |S_{D}| = \Theta(2^n)
    \]

    \vspace{0.30cm}
    最坏情况下, $|S_{D}| = \Omega(2^n)$
  \end{center}
\end{frame}
%%%%%%%%%%%%%%%%%%%%

%%%%%%%%%%%%%%%%%%%%
\begin{frame}{}
  \begin{center}
    ``长度为 $m \ge n$个字符的$a$, $b$串, 且倒数第$n$个字符是$a$''

    \pause
    \[
      L_{n} = (a | b)^{\ast} a (a | b)^{n-1}
    \]

    \pause
    \fig{width = 0.80\textwidth}{figs/nfa-Ln}

    \pause
    \vspace{0.30cm}
    \blue{\bf 练习(非作业):} $m = n = 3$
  \end{center}
\end{frame}
%%%%%%%%%%%%%%%%%%%%

%%%%%%%%%%%%%%%%%%%%
\begin{frame}{}
  \begin{center}
    \red{\bf 闭包 (Closure):} $f\text{-closure}(T)$

    \pause
    \[
      \epsilon\text{-closure}(T) \qquad \pause
      R^{+}: \text{传递闭包}
    \]

    \pause
    \[
      T \implies f(T) \implies f(f(T)) \implies f(f(f(T))) \implies \dots
    \]

    \pause
    直到找到 $x$ 使得 $f(x) = x$ ($x$ 称为 $f$ 的\red{\bf 不动点})
  \end{center}
\end{frame}
%%%%%%%%%%%%%%%%%%%%

%%%%%%%%%%%%%%%%%%%%
\begin{frame}{}
  \fig{width = 0.70\textwidth}{figs/re-dfa-lexer}

  \vspace{0.30cm}
  \begin{center}
    DFA 最小化
  \end{center}
\end{frame}
%%%%%%%%%%%%%%%%%%%%

%%%%%%%%%%%%%%%%%%%%
\begin{frame}{}
  \fig{width = 0.50\textwidth}{figs/dfa-abb}

  \[
    \blue{L(\mathcal{A}) = L((a|b)^{\ast}abb)}
  \]

  \fig{width = 0.50\textwidth}{figs/dfa-abb-subset}

  \begin{center}
    子集构造法得到的 DFA
  \end{center}
\end{frame}
%%%%%%%%%%%%%%%%%%%%

%%%%%%%%%%%%%%%%%%%%
\begin{frame}{}
  \begin{center}
    \red{\bf DFA 最小化算法}基本思想: \blue{\bf 等价}的状态可以合并

    \vspace{0.30cm}
    \fig{width = 0.30\textwidth}{figs/Hopcroft}
    \teal{John Hopcroft (1939 $\sim$)}

    \vspace{0.30cm}
    \begin{quote}
      \centering
      ``for fundamental achievements in the design and analysis \\
      of \red{\emph{\textbf{algorithms and data structures}}}'' \\
      \hfill --- Turing Award, 1986
    \end{quote}
  \end{center}
\end{frame}
%%%%%%%%%%%%%%%%%%%%

%%%%%%%%%%%%%%%%%%%%
\begin{frame}{}
  \begin{center}
    如何定义\red{\bf 等价}状态?

    \fig{width = 0.50\textwidth}{figs/dfa-abb-subset}

    \[
      s \sim t \iff \forall a \in \Sigma.\; 
        \left((s \rel{a} s') \land (t \rel{a} t')\right) \implies (\red{s' = t'}).
    \]

    \pause
    但是, 这个定义是错误的
  \end{center}
\end{frame}
%%%%%%%%%%%%%%%%%%%%

%%%%%%%%%%%%%%%%%%%%
\begin{frame}{}
  \[
    s \sim t \iff \forall a \in \Sigma.\; 
      \left((s \rel{a} s') \land (t \rel{a} t')\right) \implies (\red{s' = t'}).
  \]
  
  \fig{width = 0.50\textwidth}{figs/dfa-abb-subset}

  \vspace{-0.30cm}
  \[
    A \sim C \sim E
  \]

  \pause
  \vspace{0.20cm}
  \begin{center}
    但是, 接受状态与非接受状态必定不等价

    \vspace{0.30cm}
    空串 $\epsilon$ {\bf 区分}了这两类状态
  \end{center}
\end{frame}
%%%%%%%%%%%%%%%%%%%%

%%%%%%%%%%%%%%%%%%%%
\begin{frame}{}
  \[
    s \sim t \iff \forall a \in \Sigma.\; 
      \left((s \rel{a} s') \land (t \rel{a} t')\right) \implies (\red{s' = t'}).
  \]
  \fig{width = 0.60\textwidth}{figs/min-dfa-wiki-before}
  \[
    c \sim d \sim e \qquad \red{a \nsim b}
  \]
  \fig{width = 0.60\textwidth}{figs/min-dfa-wiki-after}
\end{frame}
%%%%%%%%%%%%%%%%%%%%

%%%%%%%%%%%%%%%%%%%%
\begin{frame}{}
  \begin{center}
    如何定义\red{\bf 等价}状态?
  \end{center}

  \fig{width = 0.50\textwidth}{figs/dfa-abb-subset}

  \[
    s \sim t \iff \forall a \in \Sigma.\; 
      \left((s \rel{a} s') \land (t \rel{a} t')\right) \implies (\red{s' \sim t'}).
  \]

  \pause
  \[
    s \nsim t \iff \exists a \in \Sigma.\;
      (s \rel{a} s') \land (t \rel{a} t') \land (\red{s' \nsim t'})
  \]
\end{frame}
%%%%%%%%%%%%%%%%%%%%

%%%%%%%%%%%%%%%%%%%%
\begin{frame}{}
  \begin{center}
    \[
      s \sim t \iff \forall a \in \Sigma.\; 
        \left((s \rel{a} s') \land (t \rel{a} t')\right) \implies (s' \sim t').
    \]

    基于该定义, 不断\red{\bf 合并}等价的状态, 直到无法合并为止

    \pause
    \vspace{0.80cm}
    \red{\bf 但是, 这是一个递归定义, 从哪里开始呢?}

    \pause
    \vspace{0.80cm}
    \blue{$Q:$ 所有接受状态都是等价的吗?}

    \pause
    \vspace{0.50cm}
    \fig{width = 0.60\textwidth}{figs/a-ab}
  \end{center}
\end{frame}
%%%%%%%%%%%%%%%%%%%%

%%%%%%%%%%%%%%%%%%%%
\begin{frame}{}
  \begin{center}
    \[
      s \sim t \iff \forall a \in \Sigma.\; 
        \left((s \rel{a} s') \land (t \rel{a} t')\right) \implies (s' \sim t').
    \]

    缺少基础情况, 不知从何下手

    \pause
    \[
      s \nsim t \iff \exists a \in \Sigma.\;
        (s \rel{a} s') \land (t \rel{a} t') \land (s' \nsim t')
    \]

    \pause
    \red{\bf 划分, 而非合并}
  \end{center}

\end{frame}
%%%%%%%%%%%%%%%%%%%%

%%%%%%%%%%%%%%%%%%%%
\begin{frame}{}
  \begin{center}
    \[
      s \nsim t \iff \exists a \in \Sigma.\;
        (s \rel{a} s') \land (t \rel{a} t') \land (s' \nsim t')
    \]

    \pause
    \vspace{0.30cm}
    \red{\bf 从哪里开始呢?}
  \end{center}

  \pause
  \begin{columns}
    \column{0.50\textwidth}
      \fig{width = 1.00\textwidth}{figs/dfa-abb-subset}
    \column{0.50\textwidth}
      \[
        \Pi = \set{F, S \setminus F}
      \]

      \vspace{0.30cm}
      \begin{center}
        接受状态与非接受状态必定不等价

        \pause
        \vspace{0.50cm}
        空串 $\epsilon$ {\bf 区分}了这两类状态
      \end{center}
  \end{columns}
\end{frame}
%%%%%%%%%%%%%%%%%%%%

%%%%%%%%%%%%%%%%%%%%
\begin{frame}{}
  \begin{center}
    \fig{width = 0.40\textwidth}{figs/blackboard}

    \vspace{0.30cm}
    板书演示算法过程
  \end{center}
\end{frame}
%%%%%%%%%%%%%%%%%%%%

%%%%%%%%%%%%%%%%%%%%
\begin{frame}{}
  \begin{align*}
    \Pi_{0} &= \set{\set{A, B, C, D}, \set{E}} \\[10pt]
    \Pi_{1} &= \set{\set{A, B, C}, \set{D}, \set{E}} \\[10pt]
    \Pi_{2} &= \set{\set{A, C}, \set{B}, \set{D}, \set{E}} \\[10pt]
    \Pi_{3} &= \Pi_{2}
  \end{align*}
\end{frame}
%%%%%%%%%%%%%%%%%%%%

%%%%%%%%%%%%%%%%%%%%
\begin{frame}{}
  \fig{width = 0.50\textwidth}{figs/dfa-abb-subset}

  \begin{center}
    \blue{\bf 合并}等价状态 $A \sim C$
  \end{center}

  \fig{width = 0.50\textwidth}{figs/dfa-abb}

  \pause
  \begin{center}
    \red{$Q:$ 合并后是否一定还是 DFA?}
  \end{center}
\end{frame}
%%%%%%%%%%%%%%%%%%%%

%%%%%%%%%%%%%%%%%%%%
\begin{frame}{}
  \begin{center}
    DFA 最小化\red{\bf 等价状态划分}方法
  \end{center}

  \[
    \Pi = \set{F, S \setminus F}
  \]
  
  \fig{width = 0.85\textwidth}{figs/partition}

  \begin{center}
    直到再也无法\blue{\bf 划分}为止 \quad (\purple{不动点!})

    \vspace{0.40cm}
    然后, 将同一等价类里的状态\blue{\bf 合并}
  \end{center}
\end{frame}
%%%%%%%%%%%%%%%%%%%%

%%%%%%%%%%%%%%%%%%%%
\begin{frame}{}
  \begin{center}
    如何分析DFA最小化算法的\red{\bf 复杂度}?

    \pause
    \vspace{0.60cm}
    为什么 DFA 最小化算法是\red{\bf 正确}的?

    \pause
    \vspace{0.60cm}
    最小化DFA是\red{\bf 唯一}的吗?

    \pause
    \vspace{0.30cm}
    \fig{width = 0.50\textwidth}{figs/report}
    \teal{可选报告 (1)}
  \end{center}
\end{frame}
%%%%%%%%%%%%%%%%%%%%

%%%%%%%%%%%%%%%%%%%%
% \begin{frame}{}
%   \begin{center}
%     如何定义\red{\bf 等价}状态?
% 
%   \end{center}
% 
%   \pause
%   \begin{definition}[等价状态]
%     两个状态是等价的当且仅当它们对任何相同输入串的行为是相同的。
%   \end{definition}
% \end{frame}
%%%%%%%%%%%%%%%%%%%%

%%%%%%%%%%%%%%%%%%%%
\begin{frame}{}
  \fig{width = 0.70\textwidth}{figs/re-dfa-lexer}

  \vspace{0.30cm}
  \begin{center}
    \blue{DFA $\implies$ RE}
    \[
      \boxed{D \implies r}
    \]
    \[
      \text{要求}: \red{L(r) = L(D)}
    \]
  \end{center}
\end{frame}
%%%%%%%%%%%%%%%%%%%%

%%%%%%%%%%%%%%%%%%%%
\begin{frame}{}
  \begin{center}
    \fig{width = 0.30\textwidth}{figs/dfa-re-wiki}

    \pause
    \vspace{-0.30cm}
    \[
      L(D) = \set{x \mid \exists f \in F_{D}.\; d_{0} \rel{x} f}
    \]
    \pause
    \vspace{-0.30cm}
    \[
      r = \red{\mid}_{x \in L(D)}\;\; x
    \]

    \pause
    \vspace{0.30cm}
    字符串 $x$ 对应于有向图中的路径

    \pause
    \vspace{0.30cm}
    求有向图中所有 (从初始状态到接受状态的) 路径

    \pause
    \vspace{0.30cm}
    \red{但是, 如果有向图中含有{\bf 环}, 则存在无穷多条路径}

    \pause
    \vspace{0.30cm}
    \blue{不要怕, 我们有 Kleene 闭包}
  \end{center}
\end{frame}
%%%%%%%%%%%%%%%%%%%%

%%%%%%%%%%%%%%%%%%%%
\begin{frame}{}
  \begin{center}
    假设有向图中节点编号为0\purple{(初始状态)}到 $n - 1$

    \vspace{0.50cm}
    $R^{k}_{ij}:$ 从节点 $i$ 到节点 $j$、且\red{\bf 中间节点}编号\red{\bf 不大于$k$}的所有路径

    \pause
    \fig{width = 0.50\textwidth}{figs/dfa-re-kleene}
    \[
      R^{k}_{ij} = R_{ik}^{k-1} \blue{(R^{k-1}_{kk})^{\ast}} R^{k-1}_{kj} 
        \mid R_{ij}^{k-1}
    \]
  \end{center}
\end{frame}
%%%%%%%%%%%%%%%%%%%%

%%%%%%%%%%%%%%%%%%%%
\begin{frame}{}
  \begin{center}
    \red{$Q:$ 如何初始化?}

    \vspace{0.30cm}
    $R^{-1}_{ij}:$ \pause 从节点 $i$ 到节点 $j$、且\red{\bf 不经过中间节点}的所有路径
  \end{center}

  \pause
  \vspace{0.30cm}
  \begin{columns}
    \column{0.50\textwidth}
      \fig{width = 0.70\textwidth}{figs/i-j}
      \[
        R^{-1}_{ij} = a_{1} | a_{2} | \dots | a_{m}
      \]

      \fig{width = 0.60\textwidth}{figs/i-j-no-edge}
      \[
        R^{-1}_{ij} = \red{\emptyset}
      \]
    \column{0.50\textwidth}
      \pause
      \fig{width = 0.60\textwidth}{figs/i-loop}
      \[
        R^{-1}_{ii} = a_{1} | a_{2} | \dots | a_{m} | \red{\epsilon}
      \]

      \fig{width = 0.15\textwidth}{figs/i}
      \[
        R^{-1}_{ii} = \red{\epsilon}
      \]
  \end{columns}
\end{frame}
%%%%%%%%%%%%%%%%%%%%

%%%%%%%%%%%%%%%%%%%%
\begin{frame}{}
  \begin{center}
    关于 \red{$\emptyset$} (注意: 它不是正则表达式) 的规定 
    \[
      \emptyset r = r \emptyset = \emptyset
    \]

    \[
      \emptyset | r = r
    \]
  \end{center}
\end{frame}
%%%%%%%%%%%%%%%%%%%%

%%%%%%%%%%%%%%%%%%%%
\begin{frame}{}
  \begin{center}
    \red{$Q:$ $r$ 的最终结果是什么?}

    \vspace{0.50cm}
    求有向图中所有 (从初始状态到接受状态的) 路径

    \pause
    \vspace{0.50cm}
    \[
      r = |_{s_{j} \in F_{D}} R^{|S_{D}| - 1}_{0j}
    \]
  \end{center}
\end{frame}
%%%%%%%%%%%%%%%%%%%%

%%%%%%%%%%%%%%%%%%%%
\begin{frame}{}
  \fig{width = 0.60\textwidth}{figs/dfa-re-alg}

  \begin{center}
    $|D|$: 状态数 ($|S_{D}|$); \qquad $D_{A}$: 接受状态集合 ($F_{D}$)
  \end{center}
\end{frame}
%%%%%%%%%%%%%%%%%%%%

%%%%%%%%%%%%%%%%%%%%
\begin{frame}{}
  \fig{width = 0.30\textwidth}{figs/dfa-re-wiki}

  \vspace{-0.50cm}
  \fig{width = 0.12\textwidth}{figs/dfa-re-init}
\end{frame}
%%%%%%%%%%%%%%%%%%%%

%%%%%%%%%%%%%%%%%%%%
\begin{frame}{}
  \fig{width = 0.30\textwidth}{figs/dfa-re-wiki}

  \vspace{-0.50cm}
  \fig{width = 0.65\textwidth}{figs/dfa-re-step-0}
\end{frame}
%%%%%%%%%%%%%%%%%%%%

%%%%%%%%%%%%%%%%%%%%
\begin{frame}{}
  \fig{width = 0.30\textwidth}{figs/dfa-re-wiki}

  \vspace{-0.50cm}
  \fig{width = 0.70\textwidth}{figs/dfa-re-step-1}
\end{frame}
%%%%%%%%%%%%%%%%%%%%

%%%%%%%%%%%%%%%%%%%%
\begin{frame}{}
  \fig{width = 0.30\textwidth}{figs/dfa-re-wiki}

  \vspace{-0.50cm}
  \fig{width = 0.95\textwidth}{figs/dfa-re-step-2}
\end{frame}
%%%%%%%%%%%%%%%%%%%%

%%%%%%%%%%%%%%%%%%%%
\begin{frame}{}
  \fig{width = 0.70\textwidth}{figs/re-dfa-lexer}

  \vspace{0.30cm}
  \begin{center}
    \blue{DFA $\implies$ 词法分析器}
  \end{center}
\end{frame}
%%%%%%%%%%%%%%%%%%%%

%%%%%%%%%%%%%%%%%%%%
\begin{frame}{}
  \fig{width = 0.70\textwidth}{figs/3-regex}
\end{frame}
%%%%%%%%%%%%%%%%%%%%

%%%%%%%%%%%%%%%%%%%%
\begin{frame}{}
  \begin{center}
    根据正则表达式构造相应的NFA
  \end{center}

  \fig{width = 0.60\textwidth}{figs/3-nfa}
\end{frame}
%%%%%%%%%%%%%%%%%%%%

%%%%%%%%%%%%%%%%%%%%
\begin{frame}{}
  \begin{center}
    合并这三个NFA, 构造 $a | abb | a^{\ast}b^{+}$ 对应的 NFA
  \end{center}

  \fig{width = 0.70\textwidth}{figs/3-nfa-into-1}
\end{frame}
%%%%%%%%%%%%%%%%%%%%

%%%%%%%%%%%%%%%%%%%%
\begin{frame}{}
  \begin{center}
    使用\blue{\bf 子集构造法}将NFA转化为等价的DFA (\teal{并消除``死状态'' $\emptyset$})

    \fig{width = 0.70\textwidth}{figs/dfa-lexer-example}

    \vspace{0.30cm}
    \teal{注意: 要保留各个NFA的\red{\bf 接受状态}信息, 并采用\red{\bf 最前优先匹配}原则}
  \end{center}
\end{frame}
%%%%%%%%%%%%%%%%%%%%

%%%%%%%%%%%%%%%%%%%%
\begin{frame}{}
  \begin{center}
    模拟运行该 DFA, \red{直到无法继续为止} (\teal{输入结束或状态无转移}); \\[3pt]
    假设此时状态为 $s$

    \fig{width = 0.70\textwidth}{figs/dfa-lexer-example}

    \pause
    若 $s$ 为\blue{\bf 接受状态}, 则识别成功;

    \vspace{0.30cm}
    否则, \blue{\bf 回溯} (\teal{包括状态与输入流})至最近一次经过的接受状态, 识别成功;

    \vspace{0.30cm}
    若没有经过任何接受状态, 则\red{\bf 报错} (\teal{忽略第一个字符})

    \vspace{0.30cm}
    无论成功还是失败, 都从{\bf 初始状态}开始继续识别下一个词法单元
  \end{center}
\end{frame}
%%%%%%%%%%%%%%%%%%%%

%%%%%%%%%%%%%%%%%%%%
\begin{frame}
  \fig{width = 0.70\textwidth}{figs/dfa-lexer-example}

  \[
    x = a \qquad \text{输入结束; 接受; 识别出 $a$}
  \]

  \pause
  \vspace{-0.40cm}
  \[
    x = abba \qquad \text{状态无转移; 接受; 识别出 $abb$}
  \]

  \pause
  \vspace{-0.40cm}
  \[
    x = aaaa \qquad \text{输入结束; 回溯成功; 识别出 $a$}
  \]

  \pause
  \vspace{-0.40cm}
  \[
    x = cabb \qquad \text{状态无转移; 回溯失败; 报错 $c$}
  \]
\end{frame}
%%%%%%%%%%%%%%%%%%%%

%%%%%%%%%%%%%%%%%%%%
\begin{frame}{}
  \begin{center}
    特定于词法分析器的DFA最小化方法

    \fig{width = 0.70\textwidth}{figs/dfa-lexer-example}

    \vspace{0.30cm}
    初始划分需要考虑不同的\red{\bf 词法单元}

    \vspace{-0.50cm}
    \[
      \Pi_{0} = \set{\set{0137, 7}, \red{\set{247}, \set{8, 58}, \set{68}}}
    \]

    \pause
    \vspace{-1.00cm}
    \[
      \Pi_{1} = \set{\blue{\set{0137}, \set{7}}, \set{247}, \blue{\set{8}, \set{58}}, 
        \set{68}}
    \]
  \end{center}
\end{frame}
%%%%%%%%%%%%%%%%%%%%