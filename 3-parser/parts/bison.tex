% bison.tex

%%%%%%%%%%%%%%%%%%%%
\begin{frame}{}
  \fig{width = 0.40\textwidth}{figs/bison}

  \vspace{0.10cm}
  \begin{center}
    \red{\bf Bison:} 语法分析器的生成器
  \end{center}
\end{frame}
%%%%%%%%%%%%%%%%%%%%

%%%%%%%%%%%%%%%%%%%%
\begin{frame}{}
  \begin{center}
    \red{\bf 多行表达式文法:} 每行都以换行符结束; 允许有空行

    \fig{width = 0.50\textwidth}{figs/bison-cfg-expr}

    \teal{\texttt{file: expr.y}}
  \end{center}
\end{frame}
%%%%%%%%%%%%%%%%%%%%

%%%%%%%%%%%%%%%%%%%%
\begin{frame}{}
  \begin{center}
    \red{\bf 产生式 + 动作:} \teal{$LALR(1)$}使用该产生式归约时执行动作

    \fig{width = 0.95\textwidth}{figs/bison-expr-actions}
  \end{center}
\end{frame}
%%%%%%%%%%%%%%%%%%%%

%%%%%%%%%%%%%%%%%%%%
\begin{frame}{}
  \begin{center}
    \red{\bf 类型:} 为每个文法符号的\blue{\bf 属性值}指定合适的类型

    \fig{width = 0.95\textwidth}{figs/bison-expr-types}
  \end{center}
\end{frame}
%%%%%%%%%%%%%%%%%%%%

%%%%%%%%%%%%%%%%%%%%
\begin{frame}{}
  \begin{center}
    \teal{\texttt{bison -d -v expr.y}}

    \vspace{0.80cm}
    \texttt{-d}: (definition) 产生 \teal{\texttt{expr.tab.h}}
      与 \teal{\texttt{expr.tab.c}}

    \vspace{0.80cm}
    \texttt{-v}: ``warning: 20 \red{\bf shift/reduce} conflicts''
      \teal{(\texttt{expr.output})}
  \end{center}
\end{frame}
%%%%%%%%%%%%%%%%%%%%

%%%%%%%%%%%%%%%%%%%%
\begin{frame}{}
  \begin{center}
    使用\red{\bf 结合性/优先级}解决\red{\bf 移入/归约}冲突

    \fig{width = 0.80\textwidth}{figs/bison-expr-assoc-pred}

    原则上, 可以为每个\blue{\bf 终结符}以及每条\blue{\bf 产生式}赋予适当的结合性与优先级

    \fig{width = 0.80\textwidth}{figs/bison-expr-UMINUS}

    \pause
    产生式的结合性与优先级被设定为它的\purple{\bf 最右终结符号}的结合性与优先级
  \end{center}
\end{frame}
%%%%%%%%%%%%%%%%%%%%

%%%%%%%%%%%%%%%%%%%%
\begin{frame}{}
  \begin{center}
    使用\red{\bf 结合性/优先级}解决\red{\bf 移入/归约}冲突

    \vspace{0.80cm}
    \blue{\bf 冲突: ``移入$a$'' \emph{vs.} ``按 $A \to \alpha$ 归约''}

    \vspace{0.80cm}
    产生式的优先级高于$a$的优先级 \\[8pt]
    \blue{或者}优先级相同但产生式是左结合的, 则选择归约
  \end{center}
\end{frame}
%%%%%%%%%%%%%%%%%%%%

%%%%%%%%%%%%%%%%%%%%
\begin{frame}{}
  \begin{center}
    \fig{width = 1.00\textwidth}{figs/bison-expr-l}

    \teal{\texttt{file: expr.l}}
  \end{center}
\end{frame}
%%%%%%%%%%%%%%%%%%%%

%%%%%%%%%%%%%%%%%%%%
\begin{frame}{}
  \begin{center}
    \fig{width = 0.65\textwidth}{figs/bison-main}

    \teal{\texttt{file: main.c}}
  \end{center}
\end{frame}
%%%%%%%%%%%%%%%%%%%%

%%%%%%%%%%%%%%%%%%%%
\begin{frame}{}
  \begin{center}
    \teal{
      \texttt{\blue{\bf bison} -d -v expr.y} \\[10pt]
      \texttt{\blue{\bf flex} expr.l} \\[10pt]
      \texttt{\blue{\bf gcc} main.c lex.yy.c expr.tab.c -lfl -o expr} \\[30pt]
      \texttt{\red{\bf ./expr} expr.cmm}
    }
  \end{center}
\end{frame}
%%%%%%%%%%%%%%%%%%%%

%%%%%%%%%%%%%%%%%%%%
\begin{frame}{}
  \begin{center}
    使用 \red{\bf 错误产生式} (Error Production) 进行\blue{\bf 错误恢复}
    \[
      A \to \text{\bf error}\; \alpha \qquad \red{\alpha \in N^{\ast}}
    \]

    \fig{width = 1.00\textwidth}{figs/bison-error}

    \pause
    \vspace{0.30cm}
    \begin{description}
      \setlength{\itemsep}{8pt}
      \item[调整状态] 不断出栈, 直到碰到某状态包含形如
        $A \to \cdot\; \text{\bf error}\; \alpha$的项; 移入{\bf error}
      \item[丢弃输入] 不断调用词法分析器, 寻找终结符号串$\alpha$并移入
      \item[假装成功] 此时栈顶为 $\text{\bf error}\; \alpha$, 归约为 $A$; 恢复正常
    \end{description}
  \end{center}
\end{frame}
%%%%%%%%%%%%%%%%%%%%