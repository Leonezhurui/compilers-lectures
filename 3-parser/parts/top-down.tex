% top-down.tex

%%%%%%%%%%%%%%%%%%%%
\begin{frame}{}
  \fig{width = 0.80\textwidth}{figs/show-me-the-code}
\end{frame}
%%%%%%%%%%%%%%%%%%%%

%%%%%%%%%%%%%%%%%%%%
\begin{frame}{}
  \fig{width = 0.90\textwidth}{figs/cfg-if-then-else-list}

  \vspace{0.30cm}
  \begin{center}
    \red{\bf 顺序语句}、条件语句、打印语句
  \end{center}
\end{frame}
%%%%%%%%%%%%%%%%%%%%

%%%%%%%%%%%%%%%%%%%%
\begin{frame}{}
  \fig{width = 0.80\textwidth}{figs/cfg-if-then-else-list-First}

  \vspace{0.60cm}
  \begin{center}
    \red{每个产生式都以一个终结符开头, 且这些终结符各不相同}

    \vspace{0.30cm}
    因此, 仅根据输入中的\blue{\bf 当前词法单元}, 就可以确定应该使用哪条产生式
  \end{center}
\end{frame}
%%%%%%%%%%%%%%%%%%%%

%%%%%%%%%%%%%%%%%%%%
\begin{frame}{}
  \fig{width = 0.80\textwidth}{figs/cfg-if-then-else-list-First}

  \vspace{0.60cm}
  \fig{width = 1.00\textwidth}{figs/ll-if-then-else-list-token}

  \vspace{0.30cm}
  \begin{center}
    \textsl{getToken}: \blue{语法分析器}\red{\bf 按需}向\blue{词法分析器}索要下一个词法单元
  \end{center}
\end{frame}
%%%%%%%%%%%%%%%%%%%%

%%%%%%%%%%%%%%%%%%%%
\begin{frame}{}
  \fig{width = 0.80\textwidth}{figs/cfg-if-then-else-list-First}

  \fig{width = 1.00\textwidth}{figs/ll-if-then-else-list-advance-eat}

  \begin{center}
    \textsl{eat(t)}: 根据当前的产生式, \red{\bf 预期}的词法单元应该是 $t$

    \vspace{0.30cm}
    \red{\bf 匹配} $t$, 继续试图匹配下一个词法单元; 否则, 报错
  \end{center}
\end{frame}
%%%%%%%%%%%%%%%%%%%%

%%%%%%%%%%%%%%%%%%%%
\begin{frame}{}
  \fig{width = 0.70\textwidth}{figs/cfg-if-then-else-list-First}

  \fig{width = 0.90\textwidth}{figs/ll-if-then-else-list-procedures}

  \begin{center}
    为每个\red{\bf 非终结符}写一个\blue{\bf 递归函数}

    \vspace{0.20cm}
    对于每个\red{\bf 产生式}, 写一个 \texttt{case} \blue{\bf 分支语句}
  \end{center}
\end{frame}
%%%%%%%%%%%%%%%%%%%%

%%%%%%%%%%%%%%%%%%%%
\begin{frame}{}
  \[
    \ifkw\; 1 = 2 \;\thenkw \;\beginkw\; \printkw\; 2 = 1; \;\endkw\; \elsekw\; 1 = 1
  \]
  \[
    \blue{\ifkw\; \num = \num \;\thenkw \;\beginkw\; \printkw\; \num = \num; \;\endkw\; \elsekw\; \num = \num}
  \]

  \pause
  \fig{width = 0.35\textwidth}{figs/blackboard}

  \begin{center}
    板书演示这个\blue{语法分析器}的工作过程
  \end{center}
\end{frame}
%%%%%%%%%%%%%%%%%%%%

%%%%%%%%%%%%%%%%%%%%
\begin{frame}{}
  \fig{width = 0.90\textwidth}{figs/ll-if-then-else-list}
\end{frame}
%%%%%%%%%%%%%%%%%%%%

%%%%%%%%%%%%%%%%%%%%
\begin{frame}{}
  \begin{center}
    \uncover<2->{
        $E$ 在\purple{\bf 不消耗任何词法单元}的情况下, 直接递归调用 $E$, 造成\green{\bf 死循环}
    }
    % cfg-expr-add-sub-mul-div-ETF.tex

\begin{empheq}[box=\widefbox]{align*}
  E &\to E + T \mid \red{E - T} \mid T \\[8pt]
  T &\to \red{T \ast F} \mid T / F \mid F \\[8pt]
  F &\to \id \mid \num \mid (E)
\end{empheq}

    \fig{width = 0.80\textwidth}{figs/ll-expr-E}

    \uncover<3->{
        \vspace{0.30cm}
        更重要的是, $E$与$T$ 的产生式所能生成的句子可能\purple{\bf 以相同的终结符开头}

        \vspace{0.10cm}
        因此, \blue{\bf 无法仅根据输入中当前的词法单元确定要使用的生成式}}
    % \pause
    % \fig{width = 0.60\textwidth}{figs/ll-expr-T}
  \end{center}
\end{frame}
%%%%%%%%%%%%%%%%%%%%

%%%%%%%%%%%%%%%%%%%%
\begin{frame}{}
  \begin{center}
    \red{\bf 消除左递归}
    % E-T-left-recursion.tex

\begin{empheq}[box=\widefbox]{align*}
  E &\to E + T \mid T
\end{empheq}

    \pause
    \vspace{0.30cm}
    % E-T-right-recursion.tex

\begin{empheq}[box=\widefbox]{align*}
  E &\to T E' \\[8pt]
  E' &\to +\; T E' \mid \epsilon
\end{empheq}
    将左递归转为\blue{\bf 右递归}
  \end{center}
\end{frame}
%%%%%%%%%%%%%%%%%%%%

%%%%%%%%%%%%%%%%%%%%
\begin{frame}{}
  % cfg-expr-add-mul.tex

\begin{empheq}[box=\widefbox]{align*}
  E \to E + E \mid E \ast E \mid (E) \mid \id
\end{empheq}

  \pause
  % cfg-expr-add-mul-no-left-recursion.tex

\begin{empheq}[box=\widefbox]{align*}
  E &\to T E' \\[8pt]
  E' &\to +\; T E' \mid \epsilon \\[8pt]
  T &\to F T' \\[8pt]
  T' &\to \ast\; F T' \mid \epsilon \\[8pt]
  F &\to (E) \mid \id
\end{empheq}
\end{frame}
%%%%%%%%%%%%%%%%%%%%

%%%%%%%%%%%%%%%%%%%%
\begin{frame}{}
\end{frame}
%%%%%%%%%%%%%%%%%%%%

%%%%%%%%%%%%%%%%%%%%
\begin{frame}{}
\end{frame}
%%%%%%%%%%%%%%%%%%%%

%%%%%%%%%%%%%%%%%%%%
\begin{frame}{}
\end{frame}
%%%%%%%%%%%%%%%%%%%%

%%%%%%%%%%%%%%%%%%%%
\begin{frame}{}
\end{frame}
%%%%%%%%%%%%%%%%%%%%

%%%%%%%%%%%%%%%%%%%%
\begin{frame}{}
\end{frame}
%%%%%%%%%%%%%%%%%%%%