% bottom-up.tex

%%%%%%%%%%%%%%%%%%%%
\begin{frame}{}
  \begin{center}
    只考虑\red{\bf 无二义性}的文法 \\[4pt]
    这意味着, 每个句子对应唯一的一棵语法分析树

    \fig{width = 0.60\textwidth}{figs/cfg-hierarchy}

    今日份主题: \red{\bf $LR(1)$ ($LR(0)$) 语法分析器}
  \end{center}
\end{frame}
%%%%%%%%%%%%%%%%%%%%

%%%%%%%%%%%%%%%%%%%%
\begin{frame}{}
  \begin{center}
    自顶向下的、\\[15pt]
    不断规约的、\\[15pt]
    基于句柄查找自动机的、\\[15pt]
    适用于\red{\bf $LR(\ast)$ 文法}的、\\[15pt]
    $LR(\ast)$ 语法分析器
  \end{center}
\end{frame}
%%%%%%%%%%%%%%%%%%%%

%%%%%%%%%%%%%%%%%%%%
\begin{frame}{}
  \begin{center}
    {\large \red{\bf 自底向上}构建语法分析树}

    \vspace{0.60cm}
    \blue{\bf 根节点}是文法的起始符号 $S$

    \vspace{1.00cm}
    \uncover<2->{
      每个\blue{\bf 中间非终结符节点}表示\purple{\bf 使用它的某条产生式进行归约}
    }

    \vspace{1.00cm}
    \blue{\bf 叶节点}是词法单元流 $w\$$ \\[8pt]
    仅包含终结符号与特殊的\teal{\bf 文件结束符$\$$}
  \end{center}
\end{frame}
%%%%%%%%%%%%%%%%%%%%

%%%%%%%%%%%%%%%%%%%%
\begin{frame}{}
  \begin{center}
    \blue{\bf 自顶向下的``推导''} 与 \red{\bf 自底向上的``归约''}

    \vspace{-0.30cm}
    \[
      \blue{E \rm T \rm T \ast F \rm T \ast \id \rm F \ast \id \rm \id \ast \id}
    \]

    \vspace{-0.50cm}
    \begin{columns}
      \column{0.50\textwidth}
        % cfg-expr-add-mul-mul-first-numbering.tex

\begin{empheq}[box=\widefbox]{align*}
  E &\to E + T \\[8pt]
  E &\to T \\[8pt]
  T &\to T \ast F \\[8pt]
  T &\to F \\[8pt]
  F &\to (E) \\[8pt]
  F &\to \id
\end{empheq}
      \column{0.50\textwidth}
        \fig{width = 0.50\textwidth}{figs/tree-expr-add-mul}

        \vspace{-0.80cm}
        \[
          w = \blue{\id \ast \id}
        \]
    \end{columns}

    \[
      \red{E \impliedby T \impliedby T \ast F \impliedby T \ast \id \impliedby F \ast \id
        \impliedby \id \ast \id}
    \]
  \end{center}
\end{frame}
%%%%%%%%%%%%%%%%%%%%

%%%%%%%%%%%%%%%%%%%%
\begin{frame}{}
  \begin{center}
    \blue{\bf ``推导'' ($A \to \alpha$)} 与 \red{\bf ``归约'' ($A \gets \alpha$)}

    \[
      S \triangleq \gamma_{0} \implies \dots
        \blue{\gamma_{i-1} \implies \gamma_{i} \implies \gamma_{r+1}}
        \implies \dots \implies r_{n} = w
    \]
    \[
      S \triangleq \gamma_{0} \impliedby \dots
        \red{\gamma_{i-1} \impliedby \gamma_{i} \impliedby \gamma_{r+1}}
        \impliedby \dots \impliedby r_{n} = w
    \]

    \vspace{0.80cm}
    自底向上语法分析器为输入构造\red{\bf 反向推导}
  \end{center}
\end{frame}
%%%%%%%%%%%%%%%%%%%%

%%%%%%%%%%%%%%%%%%%%
\begin{frame}{}
  \begin{center}
    \red{\bf $LR$ 语法分析器}

    \vspace{0.80cm}
    \begin{columns}
      \column{0.10\textwidth}
      \column{0.80\textwidth}
        \begin{description}
          \setlength{\itemsep}{15pt}
          \item[$L:$] \purple{\bf 从左向右} (Left-to-right) 扫描输入
          \item[$R:$] 构建\purple{\bf 反向} (Reverse) \purple{\bf 最右推导}
        \end{description}
      \column{0.10\textwidth}
    \end{columns}

    \vspace{0.80cm}
    ``反向最右推导''与``从左到右扫描''相一致
  \end{center}
\end{frame}
%%%%%%%%%%%%%%%%%%%%

%%%%%%%%%%%%%%%%%%%%
\begin{frame}{}
  \begin{center}
    \red{\bf $LR$ 语法分析器的状态}

    \vspace{0.80cm}
    在任意时刻, 语法分析树的\red{\bf 上边缘}与\blue{\bf 剩余的输入}构成当前句型

    \vspace{0.60cm}
    \fig{width = 0.90\textwidth}{figs/lr-tree-expr-add-mul}
    \[
      E \impliedby T \impliedby T \ast F \impliedby T \ast \id \impliedby F \ast \id
        \impliedby \id \ast \id
    \]

    \vspace{0.60cm}
    $LR$ 语法分析器使用\red{\bf 栈}存储语法分析树的\red{\bf 上边缘}

    \vspace{0.30cm}
    它包含了语法分析器目前所知的所有信息
  \end{center}
\end{frame}
%%%%%%%%%%%%%%%%%%%%

%%%%%%%%%%%%%%%%%%%%
\begin{frame}{}
  \begin{center}
    板书演示``\red{\bf 栈}''上操作

    \begin{columns}
      \column{0.50\textwidth}
        % cfg-expr-add-mul-mul-first-numbering.tex

\begin{empheq}[box=\widefbox]{align*}
  E &\to E + T \\[8pt]
  E &\to T \\[8pt]
  T &\to T \ast F \\[8pt]
  T &\to F \\[8pt]
  F &\to (E) \\[8pt]
  F &\to \id
\end{empheq}
      \column{0.50\textwidth}
        \fig{width = 0.50\textwidth}{figs/tree-expr-add-mul}

        \vspace{-0.80cm}
        \[
          w = \blue{\id \ast \id}
        \]
    \end{columns}

    \vspace{0.60cm}
    两大操作: \blue{\bf 移入输入符号} 与 \red{\bf 按产生式归约}

    \vspace{0.30cm}
    直到栈中仅剩开始符号 $S$, 且输入已结束, 则成功停止
  \end{center}
\end{frame}
%%%%%%%%%%%%%%%%%%%%

%%%%%%%%%%%%%%%%%%%%
\begin{frame}{}
  \begin{center}
    \red{\bf 基于栈的 $LR$ 语法分析器}

    \vspace{0.80cm}
    \red{$Q_{1}:$} 何时归约? \gray{(何时移入?)}

    \vspace{0.80cm}
    \red{$Q_{2}:$} 按哪条产生式进行规约?
  \end{center}
\end{frame}
%%%%%%%%%%%%%%%%%%%%

%%%%%%%%%%%%%%%%%%%%
\begin{frame}{}
  \begin{center}
    \red{\bf 基于栈的 $LR$ 语法分析器}

    \begin{columns}
      \column{0.50\textwidth}
        % cfg-expr-add-mul-mul-first-numbering.tex

\begin{empheq}[box=\widefbox]{align*}
  E &\to E + T \\[8pt]
  E &\to T \\[8pt]
  T &\to T \ast F \\[8pt]
  T &\to F \\[8pt]
  F &\to (E) \\[8pt]
  F &\to \id
\end{empheq}
      \column{0.50\textwidth}
        \fig{width = 0.60\textwidth}{figs/tree-expr-add-mul}
    \end{columns}

    \vspace{0.50cm}
    为什么第二个 $F$ 以 $T \ast F$ 整体被归约为 $T$?

    \vspace{0.30cm}
    这与\red{\bf 栈}的当前状态 ``$T \ast F$'' 相关
  \end{center}
\end{frame}
%%%%%%%%%%%%%%%%%%%%

%%%%%%%%%%%%%%%%%%%%
\begin{frame}{}
  \begin{center}
    \purple{\bf $LR(0)$ 分析表}指导$LR(0)$语法分析器

    \vspace{0.30cm}
    \fig{width = 0.60\textwidth}{figs/lr0-table-expr-add-mul}

    \vspace{0.10cm}
    在\red{\bf 当前状态(编号)}下, 面对\blue{\bf 当前文法符号}时, 该采取什么\brown{\bf 动作}

    \vspace{0.30cm}
    \purple{\action{}} 表指明动作, \purple{\goto{}} 表仅用于归约时的状态转换
  \end{center}
\end{frame}
%%%%%%%%%%%%%%%%%%%%

%%%%%%%%%%%%%%%%%%%%
\begin{frame}{}
  \begin{center}
    \fig{width = 0.60\textwidth}{figs/lr0-table-expr-add-mul}

    \vspace{0.30cm}
    % lr-actions.tex

% \usepackage{graphicx}
\begin{table}[]
  \centering
  \resizebox{0.50\textwidth}{!}{
    \renewcommand{\arraystretch}{1.2}
    \begin{tabular}{c||c}
      \hline
      $\brown{s}n$ & 移入, 并进入状态 $n$ \\ \hline
      $\brown{r}k$ & 使用 $k$ 号产生式进行归约 \\ \hline
      $\brown{g}n$ & 转换到状态 $n$ \\ \hline
      $\brown{a}$  & 成功接受, 结束 \\ \hline
      空白        & 错误  \\ \hline
    \end{tabular}}
\end{table}
  \end{center}
\end{frame}
%%%%%%%%%%%%%%%%%%%%

%%%%%%%%%%%%%%%%%%%%
\begin{frame}{}
  \begin{definition}[$LR(0)$文法]
    如果文法 $G$ 的\red{\bf $LR(0)$分析表}是\blue{\bf 无冲突}的,
    则 $G$ 是 $LR(0)$ 文法。
  \end{definition}

  \vspace{0.30cm}
  \begin{center}
    \blue{\bf 无冲突:} \action{}表中每个单元格最多只有一种动作 \\[8pt]

    \fig{width = 0.50\textwidth}{figs/lr0-table-expr-add-mul}

    \blue{\bf 两类可能的冲突:} ``移入/归约''冲突、``归约/归约''冲突
  \end{center}
\end{frame}
%%%%%%%%%%%%%%%%%%%%

%%%%%%%%%%%%%%%%%%%%
\begin{frame}{}
  \begin{center}
    再次板书演示``\red{\bf 栈}''上操作: \red{\bf 移入}与\red{\bf 归约}

    \begin{columns}
      \column{0.40\textwidth}
        % cfg-expr-add-mul-mul-first-numbering.tex

\begin{empheq}[box=\widefbox]{align*}
  E &\to E + T \\[8pt]
  E &\to T \\[8pt]
  T &\to T \ast F \\[8pt]
  T &\to F \\[8pt]
  F &\to (E) \\[8pt]
  F &\to \id
\end{empheq}
      \column{0.60\textwidth}
        \fig{width = 1.00\textwidth}{figs/lr0-table-expr-add-mul}
    \end{columns}

    \[
      w = \blue{\id \ast \id \$}
    \]

    \red{\bf 栈}中存储语法分析器的\purple{\bf 状态(编号)}, ``编码''了语法分析树的上边缘
  \end{center}
\end{frame}
%%%%%%%%%%%%%%%%%%%%

%%%%%%%%%%%%%%%%%%%%
\begin{frame}{}
  \begin{center}
    % \red{\bf $LR(\ast)$语法分析器框架}

    % lr-framework.tex

\begin{algorithm}[H]
% \caption{}
% \label{alg:S}
\begin{algorithmic}[1]
  \Procedure{\blue{$LR$}}{\null}
    \State $\Call{Push}{S, \$}$ \qquad $\Call{Push}{S, s_{0}}$

    \hStatex
    \State $t \gets \Call{\purple{next-token}}{\null}$
    \While{$1$}
      \State $s \gets \Call{Top}{S}$  \Comment{\blue{$s$一定是某个状态编号, 而不是文法符号}}

      \hStatex
      \If{\red{$\action[s, t] = s_{i}$}} \Comment{\brown{移入}}
        \State $\Call{Push}{S, t}$ \qquad $\Call{Push}{S, i}$
        \State $t \gets \Call{\purple{next-token}}{\null}$
      \ElsIf{\red{$\action[s, t] = r_{j}$}} \Comment{\brown{规约; $j: A \to \alpha$}}
        \State $2 \times |\alpha|$ 次 $\Call{Pop}{S}$
        \State $s \gets \Call{Top}{S}$  \Comment{\blue{$s$一定是某个状态编号, 而不是文法符号}}
        \State $\Call{Push}{S, A}$ \qquad $\Call{Push}{S, \cyan{\goto[s, A]}}$ \Comment{\brown{转换状态}}
      \ElsIf{\red{$\action[s, t] = a$}} \Comment{\brown{接受}}
        \State {\bf break}
      \Else
        \State $\Call{\teal{error}}{\dots}$
      \EndIf
    \EndWhile
  \EndProcedure
\end{algorithmic}
\end{algorithm}
  \end{center}
\end{frame}
%%%%%%%%%%%%%%%%%%%%

%%%%%%%%%%%%%%%%%%%%
\begin{frame}{}
  \fig{width = 0.80\textwidth}{figs/lr0-id-star-id}

  \begin{center}
    $w = \blue{\id \ast \id \$}$ 的分析过程
  \end{center}
\end{frame}
%%%%%%%%%%%%%%%%%%%%

%%%%%%%%%%%%%%%%%%%%
\begin{frame}{}
  \begin{center}
    \red{\bf 如何构造 $LR(0)$ 分析表?}

    \vspace{0.30cm}
    \fig{width = 0.60\textwidth}{figs/lr0-table-expr-add-mul}

    \vspace{0.10cm}
    在\red{\bf 当前状态(编号)}下, 面对\blue{\bf 当前文法符号}时, 该采取什么\brown{\bf 动作}
  \end{center}
\end{frame}
%%%%%%%%%%%%%%%%%%%%

%%%%%%%%%%%%%%%%%%%%
\begin{frame}{}
  \begin{center}
    \red{\bf 状态是什么? 如何跟踪状态?}

    \fig{width = 0.60\textwidth}{figs/lr0-table-expr-add-mul}

    状态是语法分析树的上边缘, 存储在栈中

    \vspace{0.30cm}
    可以用\red{\bf 自动机}跟踪状态变化 (\blue{自动机中的路径 $\Leftrightarrow$ 栈中符号/状态编号})
  \end{center}
\end{frame}
%%%%%%%%%%%%%%%%%%%%

%%%%%%%%%%%%%%%%%%%%
\begin{frame}{}
  \begin{center}
    \red{\bf 何时归约? 使用哪条产生式进行归约?}

    \fig{width = 0.60\textwidth}{figs/lr0-table-expr-add-mul}

    \blue{\bf 必要条件:} 当前状态中, 已观察到\blue{某个产生式的完整右部}

    \vspace{0.30cm}
    对于 $LR(0)$ 文法, 这是当前\purple{\bf 唯一}的选择
  \end{center}
\end{frame}
%%%%%%%%%%%%%%%%%%%%

%%%%%%%%%%%%%%%%%%%%
\begin{frame}{}
  \begin{center}
    \red{\bf 何时归约? 使用哪条产生式进行归约?}

    \begin{definition}[句柄 (Handle)]
      在输入串的(唯一)反向最右推导中, \purple{\bf 如果}下一步是逆用产生式 $A \to \alpha$
      将$\alpha$规约为$A$, 则称 $\alpha$ 是\blue{当前句型的}\red{\bf 句柄}。
    \end{definition}

    \vspace{0.50cm}
    \fig{width = 0.80\textwidth}{figs/lr-expr-handle}

    \vspace{0.30cm}
    $LR$语法分析器的关键就是高效\red{\bf 寻找每个归约步骤所使用的句柄}。
  \end{center}
\end{frame}
%%%%%%%%%%%%%%%%%%%%

%%%%%%%%%%%%%%%%%%%%
\begin{frame}{}
  \begin{center}
    \red{\bf 句柄可能在哪里?}

    \begin{theorem}
      \red{\bf 存在}一种$LR$语法分析方法, 保证\blue{\bf 句柄总是出现在栈顶}。
    \end{theorem}

    \pause
    \fig{width = 0.80\textwidth}{figs/rm-two-steps}

    \vspace{-0.30cm}
    \begin{columns}
      \column{0.50\textwidth}
        \[
          S \dstarrm \alpha Az \dstarrm \alpha\blue{\beta By}z
            \dstarrm \alpha\beta\blue{\gamma} yz
        \]
      \column{0.50\textwidth}
        \[
          S \dstarrm \alpha BxAz \dstarrm \alpha Bx\blue{y}z \dstarrm \alpha\blue{\gamma} xyz
        \]
    \end{columns}
  \end{center}
\end{frame}
%%%%%%%%%%%%%%%%%%%%

%%%%%%%%%%%%%%%%%%%%
\begin{frame}{}
  \begin{center}
    可以用\red{\bf 自动机}跟踪状态变化 \\[5pt]
    (\blue{自动机中的路径 $\Leftrightarrow$ 栈中符号/状态编号})

    \begin{theorem}
      \red{\bf 存在}一种$LR$语法分析方法, 保证\blue{\bf 句柄总是出现在栈顶}。
    \end{theorem}

    \vspace{0.80cm}
    在自动机的当前状态识别可能的句柄 (观察到的\purple{完整右部}) \\[5pt]
    (\blue{自动机的当前状态 $\Leftrightarrow$ 栈顶})
  \end{center}
\end{frame}
%%%%%%%%%%%%%%%%%%%%

%%%%%%%%%%%%%%%%%%%%
\begin{frame}{}
  \begin{center}
    $LR(0)$ \red{\bf 句柄识别有穷状态自动机} (Handle-Finding Automaton)
    \fig{width = 0.60\textwidth}{figs/lr0-automaton-expr}
  \end{center}
\end{frame}
%%%%%%%%%%%%%%%%%%%%

%%%%%%%%%%%%%%%%%%%%
\begin{frame}{}
  \begin{center}
    $LR(0)$ \red{\bf 句柄识别自动机}

    \vspace{0.80cm}
    为给定的\red{\bf 文法$G$}构造相应的句柄识别自动机

    \vspace{0.80cm}
    该自动机用于识别该\blue{\bf 文法$G$所允许的所有可能的句柄}
  \end{center}
\end{frame}
%%%%%%%%%%%%%%%%%%%%

%%%%%%%%%%%%%%%%%%%%
\begin{frame}{}
  \begin{center}
    $LR(0)$ \red{\bf 句柄识别自动机}
    \fig{width = 0.55\textwidth}{figs/lr0-automaton-expr}

    \red{\bf 状态}是什么? 状态之间如何\red{\bf 转移}?
  \end{center}
\end{frame}
%%%%%%%%%%%%%%%%%%%%

%%%%%%%%%%%%%%%%%%%%
\begin{frame}{}
  \begin{center}
    \red{\bf 状态}刻画了``当前观察到的\purple{\bf 针对所有}\blue{\bf 产生式的右部的前缀}''
  \end{center}

  \begin{definition}[$LR(0)$项 (Item)]
    文法 $G$ 的一个 \blue{\bf $LR(0)$ 项}是 $G$ 的某个产生式加上一个位于体部的\blue{\bf 点}。
  \end{definition}

  \[
    A \to XYZ
  \]
  \begin{align*}
    A &\to \cdot XYZ \\[6pt]
    A &\to X \cdot YZ \\[6pt]
    A &\to XY \cdot Z \\[6pt]
    A &\to XYZ \cdot
  \end{align*}

  \begin{center}
    (产生式 $A \epsilon$ 只有一个项 \blue{$A \to \cdot$})

    \vspace{0.30cm}
    \blue{\bf 项}指明了语法分析器已经观察到了某个产生式的哪些部分

    \vspace{0.20cm}
    \fbox{\blue{\bf 点}指示了\purple{\bf 栈顶}, 左边(与路径)是栈中内容, 右边是期望看到的文法符号}
  \end{center}
\end{frame}
%%%%%%%%%%%%%%%%%%%%

%%%%%%%%%%%%%%%%%%%%
\begin{frame}{}
  \begin{center}
    \red{\bf 状态}刻画了``当前观察到的\purple{\bf 针对所有}\blue{\bf 产生式的右部的前缀}''

    \begin{definition}[项集]
      \purple{\bf 项集}就是若干\blue{\bf 项}构成的集合。
    \end{definition}

    \vspace{0.30cm}
    因此, 句柄识别自动机的一个\red{\bf 状态}可以表示为一个\purple{\bf 项集}

    \pause
    \vspace{0.60cm}
    \begin{definition}[项集族]
      \teal{\bf 项集族}就是若干\purple{\bf 项集}构成的集合。
    \end{definition}

    \vspace{0.30cm}
    因此, 句柄识别自动机的\red{\bf 状态集}可以表示为一个\teal{\bf 项集族}
  \end{center}
\end{frame}
%%%%%%%%%%%%%%%%%%%%

%%%%%%%%%%%%%%%%%%%%
\begin{frame}{}
  \begin{center}
    $LR(0)$ \red{\bf 句柄识别自动机}
    \fig{width = 0.55\textwidth}{figs/lr0-automaton-expr}

    \blue{\bf 项、项集、项集族}
  \end{center}
\end{frame}
%%%%%%%%%%%%%%%%%%%%

%%%%%%%%%%%%%%%%%%%%
\begin{frame}{}
  \begin{center}
    \begin{definition}[增广文法 (Augmented Grammar)]
      文法 $G$ 的\red{\bf 增广文法}是在 $G$ 中加入产生式 \blue{$S' \to S$} 得到的文法。
    \end{definition}

    \vspace{0.50cm}
    \purple{\bf 目的:} 告诉语法分析器何时停止分析并接受输入符号串

    \vspace{0.80cm}
    当语法分析器\red{\bf 面对$\$$}且\red{\bf 要使用 $S' \to S$ 进行归约}时, 输入符号串被接受

    \vspace{0.20cm}
    \teal{注: 此``接受'' (输入串) 非彼``接受'' (句柄识别自动机)}
  \end{center}
\end{frame}
%%%%%%%%%%%%%%%%%%%%

%%%%%%%%%%%%%%%%%%%%
\begin{frame}{}
  \begin{center}
    $LR(0)$ \red{\bf 句柄识别自动机}
    \fig{width = 0.55\textwidth}{figs/lr0-automaton-expr}

    \teal{注: 此``接受'' (输入串) 非彼``接受'' (句柄识别自动机)}
  \end{center}
\end{frame}
%%%%%%%%%%%%%%%%%%%%

%%%%%%%%%%%%%%%%%%%%
\begin{frame}{}
  \begin{center}
    $LR(0)$ \red{\bf 句柄识别自动机}

    \vspace{0.30cm}
    \fig{width = 0.55\textwidth}{figs/dao}

    \vspace{0.30cm}
    \blue{\bf 初始状态是什么?}

    \vspace{0.30cm}
    \blue{\bf 状态之间如何转移?}
  \end{center}
\end{frame}
%%%%%%%%%%%%%%%%%%%%

%%%%%%%%%%%%%%%%%%%%
\begin{frame}{}
  \begin{center}
    \fbox{\blue{\bf 点}指示了\purple{\bf 栈顶}, 左边(与路径)是栈中内容, 右边是期望看到的文法符号}

    \begin{columns}
      \column{0.50\textwidth}
        % cfg-expr-add-mul-mul-first-aug-numbering.tex

\begin{empheq}[box=\widefbox]{align*}
  (0)\; E' &\to E \\[8pt]
  (1)\; E &\to E + T \\[8pt]
  (2)\; E &\to T \\[8pt]
  (3)\; T &\to T \ast F \\[8pt]
  (4)\; T &\to F \\[8pt]
  (5)\; F &\to (E) \\[8pt]
  (6)\; F &\to \id
\end{empheq}

      \column{0.50\textwidth}
        \fig{width = 0.55\textwidth}{figs/lr-expr-init-state}
    \end{columns}

    \[
      \fbox{$\closure(\set{\red{[E' \to \cdot E]}})$}
    \]
  \end{center}
\end{frame}
%%%%%%%%%%%%%%%%%%%%

%%%%%%%%%%%%%%%%%%%%
\begin{frame}{}
  \begin{center}
    板书演示 $LR(0)$ \red{\bf 句柄识别自动机}的构造过程
    \fig{width = 0.60\textwidth}{figs/lr0-automaton-expr}
  \end{center}
\end{frame}
%%%%%%%%%%%%%%%%%%%%

%%%%%%%%%%%%%%%%%%%%
\begin{frame}{}
  \begin{center}
    \fig{width = 0.70\textwidth}{figs/lr-closure-alg}

    \begin{align*}
      \textsc{goto}(I, \red{X}) &= \closure\Big(\Big\{
          \blue{[A \to \alpha X \cdot \beta]}
        \Big\lvert \red{[A \to \alpha \cdot X \beta]} \in I \Big\}\Big)
    \end{align*}
    \[
      (X \in N \cup T \cup \set{\$})
    \]
  \end{center}
\end{frame}
%%%%%%%%%%%%%%%%%%%%

%%%%%%%%%%%%%%%%%%%%
\begin{frame}{}
  \fig{width = 0.80\textwidth}{figs/lr-items}
\end{frame}
%%%%%%%%%%%%%%%%%%%%

%%%%%%%%%%%%%%%%%%%%
\begin{frame}{}
  \begin{center}
    $LR(0)$ \red{\bf 句柄识别自动机}
    \fig{width = 0.55\textwidth}{figs/lr0-automaton-expr}

    \blue{\bf $Q:$ 为什么是个有穷状态自动机?}
  \end{center}
\end{frame}
%%%%%%%%%%%%%%%%%%%%